%%%%%%%%%%%%%%%%%
% This is an sample CV template created using altacv.cls
% (v1.3, 10 May 2020) written by LianTze Lim (liantze@gmail.com). Now compiles with pdfLaTeX, XeLaTeX and LuaLaTeX.
%
%% It may be distributed and/or modified under the
%% conditions of the LaTeX Project Public License, either version 1.3
%% of this license or (at your option) any later version.
%% The latest version of this license is in
%%    http://www.latex-project.org/lppl.txt
%% and version 1.3 or later is part of all distributions of LaTeX
%% version 2003/12/01 or later.
%%%%%%%%%%%%%%%%

%% If you are using \orcid or academicons
%% icons, make sure you have the academicons
%% option here, and compile with XeLaTeX
%% or LuaLaTeX.
% \documentclass[10pt,a4paper,academicons]{altacv}

%% Use the "normalphoto" option if you want a normal photo instead of cropped to a circle
% \documentclass[10pt,a4paper,normalphoto]{altacv}

\documentclass[10pt,a4paper,ragged2d,withhyper]{altacv}
%% AltaCV uses the fontawesome5 and academicons fonts
%% and packages.
%% See http://texdoc.net/pkg/fontawesome5 and http://texdoc.net/pkg/academicons for full list of symbols. You MUST compile with XeLaTeX or LuaLaTeX if you want to use academicons.

% Change the page layout if you need to
\geometry{left=1.25cm,right=1.25cm,top=1.5cm,bottom=1.5cm,columnsep=1.2cm}

% The paracol package lets you typeset columns of text in parallel
\usepackage{paracol}

% Change the font if you want to, depending on whether
% you're using pdflatex or xelatex/lualatex
\ifxetexorluatex
  % If using xelatex or lualatex:
  \setmainfont{Roboto Slab}
  \setsansfont{Lato}
  \renewcommand{\familydefault}{\sfdefault}
\else
  % If using pdflatex:
  \usepackage[rm]{roboto}
  \usepackage[defaultsans]{lato}
  % \usepackage{sourcesanspro}
  \renewcommand{\familydefault}{\sfdefault}
\fi

% Change the colours if you want to
\definecolor{SlateGrey}{HTML}{2E2E2E}
\definecolor{LightGrey}{HTML}{666666}
\definecolor{DarkPastelRed}{HTML}{450808}
\definecolor{PastelRed}{HTML}{8F0D0D}
\definecolor{GoldenEarth}{HTML}{E7D192}
\colorlet{name}{black}
\colorlet{tagline}{PastelRed}
\colorlet{heading}{DarkPastelRed}
\colorlet{headingrule}{GoldenEarth}
\colorlet{subheading}{PastelRed}
\colorlet{accent}{PastelRed}
\colorlet{emphasis}{SlateGrey}
\colorlet{body}{LightGrey}

% Change some fonts, if necessary
\renewcommand{\namefont}{\Huge\rmfamily\bfseries}
\renewcommand{\personalinfofont}{\footnotesize}
\renewcommand{\cvsectionfont}{\LARGE\rmfamily\bfseries}
\renewcommand{\cvsubsectionfont}{\large\bfseries}


% Change the bullets for itemize and rating marker
% for \cvskill if you want to
\renewcommand{\itemmarker}{{\small\textbullet}}
\renewcommand{\ratingmarker}{\faCircle}

%% sample.bib contains your publications
\addbibresource{sample.bib}

\begin{document}
\name{Ezequiel Barbero}
\tagline{Full Stack Software Craftman}
%% You can add multiple photos on the left or right
%\photoR{2.8cm}{Globe_High}
% \photoL{2.5cm}{Yacht_High,Suitcase_High}

\personalinfo{%
  % Not all of these are required!
  \email{barbero.oe@gmail.com}
  %\phone{}
  %\mailaddress{Åddrésş, Street, 00000 Cóuntry}
  \location{Argentina}
  %\homepage{www.homepage.com}
  %\twitter{@twitterhandle}
  \linkedin{barbero-oe}
  \github{barbero-oe}
  %% You MUST add the academicons option to \documentclass, then compile with LuaLaTeX or XeLaTeX, if you want to use \orcid or other academicons commands.
  % \orcid{0000-0000-0000-0000}
  %% You can add your own arbtrary detail with
  %% \printinfo{symbol}{detail}[optional hyperlink prefix]
  % \printinfo{\faPaw}{Hey ho!}[https://example.com/]
  %% Or you can declare your own field with
  %% \NewInfoFiled{fieldname}{symbol}[optional hyperlink prefix] and use it:
  % \NewInfoField{gitlab}{\faGitlab}[https://gitlab.com/]
  % \gitlab{your_id}
}

\makecvheader
%% Depending on your tastes, you may want to make fonts of itemize environments slightly smaller
% \AtBeginEnvironment{itemize}{\small}

%% Set the left/right column width ratio to 6:4.
\columnratio{0.6}

% Start a 2-column paracol. Both the left and right columns will automatically
% break across pages if things get too long.
\begin{paracol}{2}
\cvsection{Experience}

\cvevent{Software Engineer III}{VeriTran}{October 2016 -- Ongoing}{Argentina}
\begin{itemize}
\item Multi-platform serialization mechanism using Protocol Buffers
\item CI/CD with GitLab and Docker
\item Code generator of Kotlin/Native in Python
\item Extend and integrate Microsoft MakeCode editor
\item NodeJS+Express back-end to transpile visual code to JS
\item Annotation Processor to edit fields pre-serialization process
\item Protocol to communicate between web-views and apps
\item Interview candidates and perform code challenges
\item Lead internal training on TDD, Clean Code, Design Patterns
\end{itemize}

\cvtag{Kotlin}
\cvtag{Java}
\cvtag{TypeScript}
\cvtag{ReactJS}
\cvtag{Spring}
\cvtag{JavaScript}
\cvtag{CSS}
\cvtag{NodeJS}
\cvtag{Express}
\cvtag{Git}
\cvtag{GitLab}
\cvtag{Python}
\cvtag{Bash}
\cvtag{Clojure}
\cvtag{Oracle}
\cvtag{Vim}
\cvtag{Docker}
\cvtag{Scrum}

\divider

\cvevent{Backend Developer}{Software America}{April 2015 -- October 2016}{Argentina}
\begin{itemize}
\item High level library to execute Store Procedures and queries
\item Visual label designer for Zebra Barcode printer
\item Visual planing and scheduling interface for production lines
\item Migrate and improve CRM system
\end{itemize}

\cvtag{C\#}
\cvtag{Visual Basic 6.0}
\cvtag{SVN}
\cvtag{TFS}
\cvtag{Git}
\cvtag{SQL Server}\\
\cvtag{Visual Studio}
\cvtag{Windows Forms}

\cvsection{Books I had read}

\printinfo{\faBook}{Clean Code}
\medskip

It was one of the firsts books I had read and the one that bumped me to continuously improve the way I work. Paying attention on how to write clear and simple code that is easily maintainable by myself and my colleagues.

\divider

%\printinfo{\faBook}{Game Programming Patterns}
%\medskip
%
%It explains concisely different ways of implementing patterns, and the different trade-offs that we can make. It allowed me to weight different factors when implementing these patterns.
%
%\divider

% \printinfo{\faBook}{Understanding the Four Rules of Simple Design}
% \medskip
%
% It solidified my idea of striving towards simple and clear design to improve the understanding of the system.
%
% \divider

\printinfo{\faBook}{The Software Craftsman}
\medskip

It inspired me on how to approach my professional life, on how to inspire my colleagues and the importance of having a deep understanding of the problems to solve.

\divider

\printinfo{\faBook}{Extreme Programming Explained: Embrace Change}
\medskip

It gave me a framework and ideas on how to improve the processes of the teams on which I work. With a continuous improvement mindset we can improve our work week by week.



%% Switch to the right column. This will now automatically move to the second
%% page if the content is too long.
\switchcolumn

\cvsection{My Life Philosophy}

\begin{quote}
``I embrace mastery as a life-long endeavor, there is always something new to learn and share.''
\end{quote}

\cvsection{About me}

I am an ardent supporter of battle-tested practices such as TDD, Pair Programming, Refactoring and Clean Code. These practices combined with a desire to deeply understand the business problems allowed me to adapt and deliver quality solutions consistently.

I value highly performant and motivated teams that are always looking for improvement opportunities.

I enjoy reading and I am continuously seeking new topics to deepen my understanding of the development process. My interests range from technical concepts, such as design patterns, design principles and programming languages to development practices and agile methodologies.

%My practical knowledge on these areas have granted me the capability to adapt and deliver quality solutions consistently.

%\cvsection{Interests}

\cvsection{Languages}

English\\
\divider

Spanish

%% Yeah I didn't spend too much time making all the
%% spacing consistent... sorry. Use \smallskip, \medskip,
%% \bigskip, \vpsace etc to make ajustments.
\medskip

\cvsection{Education}

\cvevent{Sup. Technician in Computer Systems}{Universidad Tecnológica Nacional (UTN)}{2011 -- 2014}{}

\divider

\cvevent{Electronic Technician}{EETN 2 Paula Albarracín de Sarmiento}{2010}{}

\end{paracol}


\end{document}
